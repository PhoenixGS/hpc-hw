\documentclass[UTF8,12pt]{article}
\usepackage{amssymb,amsfonts,amsmath,amsthm}
\usepackage{times}
\usepackage{graphicx}
\usepackage{cite}
\usepackage{xeCJK}
\usepackage{placeins}

\usepackage{listings}
\lstset{
 columns=fixed,
 numbers=left,
 numberstyle=\tiny\color{gray},
 frame=none,
 backgroundcolor=\color[RGB]{245,245,244},
 keywordstyle=\color[RGB]{40,40,255},
 numberstyle=\footnotesize\color{darkgray},
 commentstyle=\it\color[RGB]{0,96,96},
 stringstyle=\rmfamily\slshape\color[RGB]{128,0,0},
 showstringspaces=false,
 language=c++,
}

\usepackage{algorithm}  
\usepackage{algpseudocode}  
\usepackage{amsmath}  
\renewcommand{\algorithmicrequire}{\textbf{Input:}}
\renewcommand{\algorithmicensure}{\textbf{Output:}}

\usepackage{indentfirst}
\setlength{\parindent}{2em}

\renewcommand{\baselinestretch}{1.25}

\usepackage[a4paper]{geometry}
\geometry{verbose,
  tmargin=2cm,
  bmargin=2cm,
  lmargin=2cm,
  rmargin=2cm 
}

\usepackage{titlesec}
\titleformat{\section}[block]{\large \bfseries}{\arabic{section}}{1em}{}[]
\titleformat{\subsection}[block]{\normalsize \bfseries}{\arabic{section}.\arabic{subsection}}{1em}{}[]
\titleformat{\subsubsection}[block]{\small \mdseries}{\arabic{section}.\arabic{subsection}.\arabic{subsubsection}}{1em}{}[]
\titleformat{\paragraph}[block]{\footnotesize \bfseries}{[\arabic{paragraph}]}{1em}{}[]


\usepackage[x11names]{xcolor}
\usepackage{graphicx}
\usepackage{pstricks,pst-plot,pst-eps}
\usepackage{subfig}
\def\pgfsysdriver{pgfsys-dvipdfmx.def}
\usepackage{tikz}

\usepackage{verbatim}

\usepackage[colorlinks,linkcolor=red]{hyperref}
\usepackage{tabularx}


\begin{document}

\title{\bf{Exp 0报告}}
\author{方科晨\hspace{1cm}2021013400}
\date{}
\maketitle


\section{代码}
\subsection{openmp\_pow}
修改后的pow\_a函数的源代码如下:
{\setmainfont{Courier New Bold}
  \begin{lstlisting}
void pow_a(int *a, int *b, int n, int m) {
    // TODO: 使用 omp parallel for 并行这个循环
	#pragma omp parallel for
    for (int i = 0; i < n; i++) {
	int x = 1;
	for (int j = 0; j < m; j++)
	    x *= a[i];
	b[i] = x;
    }
}
  \end{lstlisting}
}

\subsection{mpi\_pow}
修改后的pow\_a函数的源代码如下:
{\setmainfont{Courier New Bold}
  \begin{lstlisting}
void pow_a(int *a, int *b, int n, int m, int comm_sz /* 总进程数 */) {
    // TODO: 对这个进程拥有的数据计算 b[i] = a[i]^m
	for (int i = 0; i < n / comm_sz; i++)
	{
		int x = 1;
		for (int j = 0; j < m; j++)
		{
			x *= a[i];
		}
		b[i] = x;
	}
}
  \end{lstlisting}
}
\section{openmp版本性能}
在 $n=112000,m=100000$ 下,使用 $1,7,14,28$ 线程的运行时间分别是: $T_1=14015212 \mu s,T_7=2020439 \mu s,T_{14}=1021353 \mu s,T_{28}=522389 \mu s$\\
可以求得相对于单线程的加速比分别为: $S(1)=1,S(7)=6.9367,S(14)=13.7222,S(28)=26.8291$ ,可以看出非常接近线性加速。

\section{mpi版本性能}
在 $n=112000,m=100000$ 下,使用 $1\times 1, 1\times 7, 1\times 14, 1\times 28, 2\times 28$ 进程下运行时间分别为 $T_1=14017820 \mu s,T_7=2012535 \mu s,T_{14}=1006988 \mu s,T_{28}=502862 \mu s,T_{56}=403601 \mu s$\\
可以求得相对于单线程的加速比分别为: $S(1)=1,S(7)=6.9653,S(14)=13.9205,S(28)=27.8761,S(56)=34.7319$ ,可以看出,在单机上,加速比可以近似线性加速,但在多机上,由于数据交换等原因,速度的增加幅度明显小于进程数的增加幅度。
\end{document}
